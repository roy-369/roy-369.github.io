\documentclass[11pt]{article}
\usepackage[margin=3cm]{geometry}
\geometry{margin=1in, headsep=0.25in}
\usepackage{amsmath}
\usepackage{amsthm}
\usepackage{amssymb}
\usepackage{physics}
\usepackage{graphicx}
\newtheorem{thm}{Theorem}
\newtheorem{lem}[thm]{Lemma}
\newtheorem{cor}{Corollary}[thm]
\theoremstyle{remark}
\newtheorem*{remark}{Remark}
\theoremstyle{definition}
\newtheorem{eg}{Example}[section]
\newtheorem{definition}{Definition}[section]
\usepackage{bbold}
\usepackage{hyperref}
\usepackage[usenames,dvipsnames]{xcolor}
\usepackage{tikz}
\usetikzlibrary{arrows.meta, automata, positioning, quotes, shapes}
\hypersetup{
	colorlinks=true,
	linkcolor=blue,
	filecolor=magenta,
	urlcolor=cyan,
}
\urlstyle{same}
\usepackage[sort&compress,numbers]{natbib}
\bibliographystyle{naturemag}
\usepackage{doi}
\newcommand{\todo}[1]{\textcolor{red}{TODO: #1}}
\title{Basics about Entanglement}
\author{Shi Feng}
\date{\today}
\begin{document}
\maketitle
\begin{thm}
	Let $\ket{\psi}$ be a pure quantum state of a lattice Hamiltonian. For an operator $O$ that is supported on one part of the bipartited of lattice the following holds: 
	\begin{equation}
		\mel{\psi}{O}{\psi} = \Tr(\rho_{i}O)
	\end{equation}
	where $\rho_{i}$ is the (reduced) density matrix of the selected part of lattice, and the trace is taken over either states thereof or of the full system (they give the same result).
\end{thm}
\begin{proof}
	The full wavefunction can be decomposed into
	\begin{equation}
		\ket{\psi} = \sum_{i,o} \omega_{i,o} \ket{\phi_i}\ket{\phi_o} 
	\end{equation}
	and $\rho_i$ is
	\begin{equation}
		\begin{split}	
			\rho_i &= \Tr_o(\ket{\psi}\bra{\psi}) = \sum_{o} \braket{\phi_o}{\psi}\braket{\psi}{\phi_o} \\
			       &=  \sum_{i,i'} \left( \sum_{o}\omega_{i,o}^{\phantom *}\:\omega_{o,i'}^* \right) \ket{\phi_i}\bra{\phi_{i'}}
		\end{split}
	\end{equation}
	Measurement tracing density matrix over all $\ket{\psi_i}$ is 
	\begin{equation}
		\begin{split}
			\Tr(\rho_i O) &= \sum_{i} \mel{\phi_i}{\rho_i O}{\phi_i}  \\
				      &= \sum_{i,i'} \left( \sum_{o} \omega_{i,o}^{\phantom *}\: \omega_{o,i'}^* \right)  \mel{\phi_{i'}}{O}{\phi_i}
		\end{split}
	\end{equation}
	Note that one can also do a trace over states of the full lattice as long as $\ket{\phi_o}$ is normalized, which doesn't matter since $O$ acting only on  $\ket{\phi_i}$.
	On the other hand, a direct meaure by $\mel{\psi}{O}{\psi}$ is 
	\begin{equation}
		\begin{split}
			\mel{\psi}{O}{\psi} &= \sum_{i',o'} \sum_{i,o} \omega_{o'i'}^* \omega_{i,o}^{\phantom *} \bra{\phi_{o'} \phi_{i'}}O\ket{\phi_i\phi_o} \\
				      &= \sum_{i,i'} \left( \sum_{o} \omega_{i,o}^{\phantom *}\: \omega_{o,i'}^* \right)  \mel{\phi_{i'}}{O}{\phi_i}
		\end{split}
	\end{equation}
	where in the last contraction we used the fact that $O$ is supported on part $i$ of Hilbert space.
\end{proof}



\begin{thm} \label{thm5}
	If $\rho$ and $\rho'$ act on disjoint Hilbert spaces, then the von-Neumann entanglement entropy satisfies
	 \[
		 S(\rho \otimes \rho') = S(\rho) + S(\rho')
	\] 
\end{thm}
\begin{proof}
	 Since entanglement is invariant under change of basis, we can diagonalize $\rho$ and $\rho'$ and calculate EE using their diagonal elements:
	  \[
		  \tilde{\rho} = \sum_{i} a_i \ket{a_i}\bra{a_i},\;\; \tilde{\rho}' = \sum_{j} b_j \ket{b_j} \bra{b_j} 
	 \] 
	 so the compoiste matrix used in $S(\rho \otimes \rho')$ is
	  \[
		  \tilde{\rho} \otimes \tilde{\rho}' = \sum_{ij} a_i b_j \ket{a_i b_j}\bra{b_j a_i} 
	 \]
	 \[
		 \log(\tilde{\rho}\otimes \tilde{\rho}') = \sum_{ij}(\log a_i + \log b_j) \ket{a_i b_j}\bra{b_j a_i} 
	 \] 
	the total entanglement entropy is thus
	 \[
	 \begin{split}	
		 S(\tilde{\rho}\otimes \tilde{\rho}') &= - \sum_{ij} a_i b_j \log(a_i) + a_i b_j \log(b_i) \\
						      &= -\sum_{i}a_i \log(a_i) \sum_{j}b_j - \sum_{j}b_j \log(b_j) \sum_{i}a_i\\
						      &= -\sum_{i}a_i\log(a_i) - b_i \log(b_i)\\
						      &= S(\tilde{\rho}) + S(\tilde{\rho}')
	 \end{split}
	 \] 
	 where we used $\Tr(\rho) = 1$ in the 2nd step. 
\end{proof}





\begin{thm}
	von-Neumann entanglement entropy is maximal iff there exists a basis in which its (reduced) density matrix is an equally weighted diagonal matrix. 
\end{thm}
\begin{proof}
	In a diagonalized reduced density matrix $diag(\epsilon_1, \ldots \epsilon_n)$, the von-Neumann entropy is
	\[
	S = -\sum_{i} \epsilon_i \log \epsilon_i\;\;\;S.t.\; \sum_{i}\epsilon_i = 1 
	\] 
To see its maximum apply Lagrangian multiplier
\[
	L(\vec{\epsilon},\lambda) = \sum_{i} -\epsilon_i \log \epsilon_i - \lambda \left( \sum_{i} \epsilon_i - 1 \right) 
\] 
by derivative wrt $\epsilon_j$ we have
\[
\forall \epsilon_i,\;\;\;\log \epsilon_i + 1 + \lambda = 0
\] 
there's only one constraint characterized by $\lambda$, so all $\epsilon_i$ must have the same value.
\end{proof}




\begin{definition}[Maximally entangled state]
	For a bipartite system $A \cup B$, the wavefunction of a maximally entangled state is given by
	 \begin{equation}
		 \ket{\psi} =  \frac{1}{\sqrt{D}} \sum_{i=1}^{D} \ket{i_A} \otimes \ket{i_B}
	\end{equation}
	with $D = \min(\dim\mathcal{H}_A, \dim \mathcal{H}_B)$,  $\braket{i}{i'} = \delta_{i,i'}$. 
\end{definition}
The interpretation as maximally entangled comes from the fact that
\[
	\rho_A = \frac{1}{D_A} \mathbb{1}_A
\] 
where we assumed $D = D_A$ WLOG. As a result, we must have
\[
	S(\rho_A) = \sum_{i} \frac{1}{D_A}\log D_A  = \log D_A \sim \text{vol }A
\] 
where $\log D_A \sim \text{vol }A$ is because (assuming 1 qubit per site)
 \[
	 \log D_A = \log(2^{\text{number of sites in A}}) \propto \text{number of sites in A}
\] 
which is proptional to the volumn of $A$ in a system where dofs are uniformly distributed.  
This is the famous bound on EE:
\begin{equation}
	0 \le S(\rho_A) \le \log D_A \propto \text{vol } A
\end{equation}



%\nocite{*}
%\printbibliography
%\bibliography{references.bib}
\end{document}
